\thispagestyle{empty}
%\hfill
\vspace*{3cm}
\noindent\Huge\textsc{Preface}\\
\normalsize
\noindent\rule[2pt]{\textwidth}{0.8pt}
\hspace*{3cm}

 This thesis was submitted to the Faculty of Science, University of Nantes, as a requirement to obtain the PhD degree. The work presented was carried out in the years 2013-2016 in the laboratory LINA ({\it Laboratoire d'Informatique de Loire Atlantique}) as part of the Inria-TASC research team, under the supervision of \'Eric {\sc MONFROY} and Florian {\sc Richoux}, and with a French Ministry of Research's Grant. I've always enjoyed my time at Nantes where I spent about two years months. I additionally spent some days \new{uslises}
 
 During this time I have participated in some schools \new{schools} and I have received some curses \new{like}.\\
 
 \Large\textsc{Context}\normalsize
 
 The evolution of computer architecture is leading us toward massively multi-core computers for tomorrow, composed of thousands of computing units. However nowadays, we do not know how to design algorithms able to manage efficiently such a computing power. In particular, this is true for combinatorial optimization algorithms, like algorithms solving constraint-based problems.
 
 There exists several techniques for solving constraint-based problems: constraint programming, linear programming, boolean satisfaction methods, and local search methods to give an non-exhaustive list. The latter is often among the most efficient techniques to solve large size problems. Nowadays and up to our knowledge, there exists only one algorithm showing very good performances scaling to thousands of cores. However its parallel scheme does not include inter-processes cooperative communications. Moreover, the rising of more and more complex algorithms leads to an number of parameters which become intractable to manage by hand, and parallel algorithms emphasize this trend.
 
 \Large\textsc{Thesis objectives}\normalsize
 
 In this context, this Ph.D. topic has two major objectives : 
 \begin{enumerate}
 	\item \new{Obj 1}
 	
 	\item \new{Obj 2}
 \end{enumerate}
 
 \Large\textsc{Problem presentation}\normalsize
 
 Optimization Problems are classical problems in Applied Mathematics and Computer Science. Their main goal is to find the best solution to a given mathematical model, which can have restrictions (constraints) or not. Variables composing the problem take their value from continuous or discrete domains. In the latter case, we are talking about of Combinatorial Optimization. Combinatorial Optimization has important applications in several fields, including machine learning, artificial intelligence, and software engineering. For example, some common problems involving combinatorial optimization are the traveling salesman problem (TSP) and the minimum spanning tree problem (MST). In some cases (problems), the main goal is only to find one solution, not the best, which is the case of the Constraint Satisfaction Problems (CSP), i.e., a solution will be a configuration of variables that complies with the constraints set. In other words: finding one feasible solution is enough.
 
 The CSPs are one of the most trendiest in the Combinatorial Optimization field, and they find a lot of application in the industry. In theoretical terms, it means that new and more complex (also bigger) problems appear. For that reason, a lot of techniques and methods are applied to the resolution of these problems. Although many of these techniques, like meta-heuristics, have shown themselves to be effective, sometimes the real problems we want to solve are too large, i.e., the search space is huge, and in most cases too long execution time is needed to find a
 solution. However, the development of the super-computing has opened a new way to find solutions for these problems in a more feasible manner, reducing the search times. Therefore, in this thesis we will focus in finding new technologies and methods for the solution of CSPs through parallel computing, developing cooperative parallel algorithms and applying auto-tuning techniques to choose the proper parameters for them, which is a field not explored yet.
 
\newpage
\Large\textsc{Thesis outline}\normalsize

%This thesis is in the form of a synopsis with attached published papers. While Part I of the thesis provides background information and an overview of the attached papers Part II constitutes the papers.

\new{In Part I, }
%\vspace*{3cm}
%\begin{center}
%%\rule{5cm}{0.2mm}\\
%Søren Preus
%\end{center}
