La technologie multi-cœur et les architectures massivement parallèles sont de plus en plus accessibles à tous, à travers des matériaux comme le Xeon Phi ou les cartes GPU. Cette stratégie d'architecture a été communément adoptée par les producteurs pour faire face à la loi de Moore. Or, ces nouvelles architectures impliquent d'autres manières de concevoir et d'implémenter les algorithmes, pour exploiter complètement leur potentiel, en particulier dans le cas des solveurs de contraintes traitant de problèmes d'optimisation combinatoire. Dans cette travail nous presentons un Langage pour créer des Solveurs Orienté Parallèle (\posl{} pour Parallel-Oriented Solver Language, et prononcé "puzzle") : cadre permettant de construire des solveurs basés sur des méta-heuristiques interconnectées travaillant en parallèle, dans le but de résoudre des instances des CSP et de mesurer sa performance. 

La nouveauté de cette approche porte sur le fait que l'on voit un solveur comme un ensemble de composants spécifiques, écris dans un langage orienté parallèle basé sur des opérateurs. Un avantage de POSL est la possibilité de partager non seulement des informations mais aussi des comportements, permettant ainsi de modifier à chaud les solveurs. POSL permets aux composants d'un solveur d'être transmis et exécutés par d'autres solveurs. Il propose également une couche supplémentaire permettant de définir dynamiquement des connexions entre solveurs.

Nous testons plusieurs stratégies de résolution, grâce au langage orienté parallèle, basé sur des opérateurs, que \posl{} fournis.