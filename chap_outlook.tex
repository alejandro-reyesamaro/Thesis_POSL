\chapter{Conclusions and Outlook}
\label{chap:Methodologies}
\textit{This chapter concludes and discuss future development within quantitative fluorescence probing of biomolecular structures}
\vspace{2ex}\vfill
\minitoc
\newpage

\section{Summary}
 The work presented in this thesis aimed at developing new tools based on fluorescence for studying the structure and dynamics of nucleic acids. In particular, quantitative FRET-based methods relying on orientationally constrained nucleobase analogues were developed. In addition, a library of fluorescent nucleobase analogue candidates was constructed which served as a source of inspiration for developing new and improved fluorescent DNA-mimicking probes.

 \paragraph{Appended papers.} The appended papers I-IV concern the development of quantitative FRET-based techniques. Paper I reviews the field of quantitative FRET with particular focus on recent advancements. Paper II reports the characterization of nucleobase analogue FRET acceptor tC$_\mathrm{nitro}$ and it was found that the lowest energy absorption band of tC$_\mathrm{nitro}$ is the result of a single electronic transition with an in-plane transition dipole moment oriented towards the nitro-group. Paper III describes a new generic methodology for the simulation and analysis of FRET in nucleic acids and demonstrates the power of the method to model base-base FRET. In Paper IV a DNA-based switch with five discrete states and a fluorescence readout is presented, where FRETmatrix was used to model the theoretical FRET efficiencies in the different states.

 Paper V and VI are focused on the characterization and development of fluorescent nucleobase analogues. In Paper V we obtain insight into the excited state decay processes of tC, tC$^\mathrm{O}$ and tC$_\mathrm{nitro}$ and explain why tC$_\mathrm{nitro}$ is non-fluorescent while tC and tC$^\mathrm{O}$ are highly fluorescent. In Paper VI we report a new fluorescent adenine analogue and describe its photophysical and base-mimicking properties in DNA.

 \paragraph{Published software.} Four general software packages for the simulation and analysis of data from spectroscopic measurements are presented. All of the programs are compiled into stand-alone executables, however, an important consideration is that they are all MATLAB-based which makes them extendable by end-users:

 FRETmatrix is a software for the simulation and analysis of FRET in highly constrained nucleic acid geometries and was published with Paper III. The software implements a new methodology based on the ability to construct any three-dimensional nucleic acid model followed by a FRET simulation between two dyes positioned in the modelled structure. FluorFit is a general time-resolved intensity decay fitting software which contains both everyday reconvolution features and more specialized capabilities, such as time-resolved FRET and global fitting. AniFit is the time-resolved fluorescence anisotropy counterpart to FluorFit. AniFit can be used to analyse multiple fluorescence anisotropy decays, both locally and globally, using a modified version of global iterative vector reconvolution. a|e is a software for the plotting, editing and analysis of UV-Vis absorption and emission spectra. Since a|e is designed specifically for spectral analysis it has a number of convenient shortcuts compared to more general data analysis and graphing software like IGOR Pro and Origin. Such features include subtraction of background scatter in absorption measurements, calculation of spectral overlap integrals in FRET, calculation of fluorescence quantum yields, singular value decomposition analysis, decomposition of spectra resulting from multiple absorbing or emitting species, and mathematical operations independent of the wavelength grids of the individual spectra.

\section{Outlook}
 Based on the findings of this thesis and my experience working in the covered areas these are my thoughts on the future development within fluorescent nucleobases and quantitative FRET.

\subsection{Novel Fluorescent Base Analogues}
 Fluorescent probes are traditionally developed based on a trial-and-error strategy, often inspired by known dye structures. However, the work reported in this thesis has shown that it is possible to rationally design molecules with DNA base-mimicking properties and at the same time to provide good predictions of their optical properties. Since high level quantum chemical methods like DFT and TDDFT are now being performed on standard laptop computers on an everyday basis, in principle without any requirements of programming skills or deep theoretical insight, this paves the way for a fundamentally new strategic approach for developing state-of-the-art fluorescent probes. The nucleobase analogue library initiated in Appendix 3 is meant to provide a source of inspiration in the development of new and improved probes, such as red-shifted excitation energies, higher brightness and higher stability. The omnipresent limitation of this design-based strategy, however, is the synthetic difficulties involved in making the most promising candidates.

 In my eyes, the holy grail in the field of fluorescent nucleobase analogues is the development of probes sufficiently bright and photostable for single-molecule detection. The combination of single-molecule FRET and fluorescent nucleobase analogues could facilitate completely new possibilities to investigate the three-dimensional structure, dynamics and thermodynamics of nucleic acids with unprecedented temporal and spatial resolution. This, however, is a more complicated task than originally imagined: Single molecule setups usually require excitation sources located well into the visible region, however, such excitation energies ideally require large $\pi$-conjugated systems thus compromising the nucleobase-mimicking properties of the probe, in particular if the probe is to be positioned in sterically demanding biological environments. It is likely that fluorescent DNA-modifications for single-molecule applications will become a new technology in the near future, however, there will probably always be a compromise between the bio-mimicking properties of the probes and their fluorescent properties.

\subsection{Quantitative FRET}
 While quantitative FRET has served as a molecular ruler for decades, the recent progress in this technique has shown that this field is only at its beginning. The ability to design experiments with multiple well-defined donor-acceptor pairs positioned at various sites within a biomolecule opens up new possibilities to probe the detailed three-dimensional position and geometry of unknown structural elements. This technique will be further aided by the well-defined positions and orientations of the base probes utilized in this thesis. Quantitative FRET-based methods are and continue to be important alternatives to more complex, time-consuming, expensive and less sensitive higher resolution techniques like biomolecular NMR spectroscopy and X-ray crystallography.

\subsubsection{The Outlook of FRETmatrix}
 Since FRETmatrix constitutes a general framework for simulating FRET in nucleic acids it is anticipated that the methodology will serve as a scaffold for even more sophisticated studies of nucleic acids structure and dynamics.

 An interesting potential extension to FRETmatrix is the probing of nucleic acid dynamics occurring on timescales longer than the lifetimes of the probes themselves, i.e. in the $\mu$s-ms timescales. Dynamics occurring in these timescales are, for example, the conformational dynamics across helix-junction-helix motifs in large functional RNAs, which are highly important in the folding of RNAs as well as for the transition between different functional states.\cite{Dethoff2012} Insight into this type of dynamics is a rapidly progressing field and currently relies on state-of-the-art NMR with residual dipolar couplings.\cite{Bothe2011} Using FRETmatrix in combination with base-base FRET local base-dynamics occurring on pico- and subnanosecond timescales may be separated from longer time-scale dynamics by proper fluorescence decay analysis. This is theoretically achievable because fast probe dynamics, such as those probed in demonstration study 1 of Paper III, leads to dynamic averaging while slow dynamics leads to static averaging of $\kappa^2$ and $R$. Dynamic averaging, in turn, leads to a single-exponential decay while static averaging leads to a multi-exponential decay, and these two types of dynamics can therefore be distinguished in the analysis of fluorescence decays. In other words, if the donor decay in a FRET experiment is single-exponential then the averaging regime is dynamic while if the decay is multi-exponential the averaging regime is static. This is of course true only when the donor decay is single-exponential in absence of acceptor, which is the case for tC$^\mathrm{O}$ and tC in double-stranded DNA. In demonstration study 1 of Paper III, the measured donor decays of tC$^\mathrm{O}$ in the presence of tC$_\mathrm{nitro}$ were well fitted using a single lifetime as a result of the rigidity of B-DNA combined with the fast timescales of local rotational dynamics of the probes which leads to dynamic averaging. This experimental observation of single-exponential donor decays in the presence of FRET supports the suggested proposal.

 A final note: FRETmatrix simulates signals, i.e. fluorescence decays in the case of Paper III, from three-dimensional nucleic acid geometrical models. It is thus equally imaginable to combine FRETmatrix with the simulation and analysis of alternative experimental techniques such as single-molecule FRET, FRET microscopy or Fluorescence Lifetime Imaging (FLIM).
